\documentclass[11pt]{article}
\usepackage[T1]{fontenc}
\usepackage[utf8]{inputenc}
\usepackage[english]{babel}
\usepackage{amsmath,amssymb}
\usepackage{url}

\usepackage{pl-judgments}

\title{\texttt{pl-judgments} Package%
\footnote{\copyright{} \the\year{} Robert Harper.  All Rights Reserved.}}
\author{Robert Harper}
\date{\today}

\begin{document}

\maketitle{}

\section*{Overview}

Include the declaration \verb|\usepackage{pl-judgments}| to define macros for the judgment forms that are typically used when defining programming languages.\footnote{The former package name, \texttt{pfpl-judgments}, is denigrated, but may still be used.}  See the source code for this document for the names of the macros and their use.

\begin{figure}[tp]
    \begin{displaymath}
        \begin{array}{l@{\qquad} l@{\qquad}l}
            \text{Judgment}   & \text{Meaning}  & \text{Invocation} \\
            \isTp{\tau}            & \tau\ \text{is a type} & \verb|\isTp{\tau}| \\
            \eqTp{\tau}{\tau'}     & \tau\ \text{and}\ \tau'\ \text{are equal types} & \verb|\eqTp{\tau}{\tau'}| \\
            \subTp{\tau}{\tau'}    & \tau\ \text{is a subtype of}\ \tau' & \verb|\subTp{\tau}{\tau'}| \\\\
            \isKd{\kappa}          & \kappa\ \text{is a kind} & \verb|\isKd{\kappa}| \\
            \eqKd{\kappa}{\kappa'} & \kappa\ \text{and}\ \kappa'\ \text{are equal kinds} & \verb|\eqKd{\kappa}{\kappa'}| \\
            \subKd{\kappa}{\kappa'} & \kappa\ \text{is a subkind of}\ \kappa' & \verb|\subKd{\kappa}{\kappa'}| \\\\
            \isOfKd{c}{\kappa}   & c\ \text{has kind}\ \kappa & \verb|\isOfKd*{c}{\kappa}|, \verb|\ofKd*{c}{\kappa}| \\
            \eqOfKd{c}{c'}{\kappa} & c\ \text{and}\ c'\ \text{are equal at kind}\ \kappa & \verb|\eqOfKd{c}{c'}{\kappa}| \\\\
            \isOfTp{e}{\tau}         & e\ \text{has type}\ \tau & \verb|\isOfTp*{e}{\tau}|, \\
            & & \verb|\isOf*{e}{\tau}|, \verb|\ofTp*{e}{\tau}| \\
            \retsTp{m}{\tau}       & m\ \text{returns type}\ \tau & \verb|\retsTp{m}{\tau}| \\
            \accsTp{k}{\tau}       & k\ \text{accepts type}\ \tau  & \verb|\accsTp{k}{\tau}| \\\\
            \hasTp{a}{\tau}        & a\ \text{has associated type}\ \tau & \verb|\hasTp{a}{\tau}| \\\\
            \isCtx{\Gamma}         & \Gamma\ \text{is a valid context} & \verb|\isCtx{\Gamma}| \\
            \isSig{\Sigma}         & \Sigma\ \text{is a valid signature} & \verb|\isSig{\Sigma}|
        \end{array}
    \end{displaymath}

    \caption{Judgments for Statics}
    \label{fig:statics}
\end{figure}

\section*{Statics}
\label{sec:statics}

The basic judgments of a statics consist of type formation and equality judgments, element and equality of elements judgments, and may distinguish computations, which may have effects, from values, which may not.  These are enriched to hypothetical/general judgments of the form $\Gamma\entails J$, written \verb|\Gamma\entails J|, where $J$ is a basic judgment, and $\Gamma$ is a \emph{context}, a collection of judgments, $j$, governing variables.  Contexts are inductively generated from the empty context, written \verb|\empCtx|, by context extension, written \verb|\extCtx{\Gamma}{j}|, to obtain $\extCtx{\Gamma}{j}$.  The concatenation of two contexts is written \verb|\appCtx{\Gamma_1}{\Gamma_2}| to obtain $\appCtx{\Gamma_1}{\Gamma_2}$, and is defined in the evident way.  Variables are invariably given meaning by \emph{substitution} of an expression for a variable to obtain an instance of a judgment; because the same thing may be substituted for two different variables, it is not meaningful to admit a disequality judgment for them, it not being stable under substitution.

A statics is sometimes parameterized by a \emph{signature}, which associates types to \emph{symbols}.  Signatures are generated from the empty signature, \verb|\empSig|, by signature extension, \verb|\extSig|, and may be combined using \verb|\appSig|, much in the manner of a context.  However, unlike variables, symbols are meaningful as such, rather than interpreted by substitution, and thereby admit a disequality judgment that is stable under bijective renamings---$a\not=b$ iff $a'\not=b'$, where the primes indicate such a renaming.

\smallskip

A collection of formation and membership judgments is given in Figure~\ref{fig:statics}.  The binary classification judgments have ``starred'' forms for declaring variables for which the membership symbol is typeset slightly tighter (as a binary operation, rather than as a relation) for use in a context or signature.  Those that admit a starred form are marked as such in the usage column; the star should be omitted when used relationally.

\begin{figure}[tp]
    \begin{displaymath}
        \begin{array}{l@{\qquad}l@{\qquad}l}
            \text{Judgment} & \text{Meaning} & \text{Invocation} \\
            \okSt[\Sigma]{s}       & s\ \text{is an ok state} & \verb|\okSt[\Sigma]{s}| \\
            \iniSt[\Sigma]{s}            & s\ \text{is initial}\ (\text{rel.} \Sigma) & \verb|\iniSt[\Sigma]{s}|, \verb|\isIni[\Sigma]{s}| \\
            \finSt[\Sigma]{s}            & s\ \text{is final}\ (\text{rel.} \Sigma) & \verb|\finSt[\Sigma]{s}|, \verb|\isFin[\Sigma]{s}| \\\\
            %
            s\stepsTo[\Sigma] s' & s\ \text{steps to}\ s'\ (\text{rel.}\ \Sigma) & \verb|s\stepsTo[\Sigma] s'| \\
            s\stepsTo[\Sigma](k) s' & s\ \text{steps to}\ s'\ \text{in}\ k\ \text{steps} & \verb|s\stepsTo[\Sigma](k) s'| \\
            s\stepsTo[\Sigma](\ast) s' & s\ \text{steps to}\ s'\ \text{in zero or more steps} & \verb|s\stepsTo[\Sigma](\ast)| \\\\
            %
            s\stepsTo[\Sigma]<\alpha> s' & s\ \text{steps to}\ s'\ \text{with action}\ \alpha\ (\text{rel.}\ \Sigma) & \verb|s\stepsTo[\Sigma]<\alpha> s'| \\\\
            %
            \isVal[\Sigma]{e}    & e\ \text{is a value}\ (\text{rel.}\ \Sigma) & \verb|\isVal[\Sigma]{e}| \\\\
            e\evalsTo[\Sigma] v          & e\ \text{evaluates to}\ v\ (\text{rel.}\ \Sigma) & \verb|e\evalsTo[\Sigma] v| \\
            e\evalsTo[\Sigma](c) v       & e\ \text{evaluates to}\ v\ \text{with cost}\ c & \verb|e\evalsTo[\Sigma](c) v| \\\\
            %
            e\raisesUp[\Sigma] v           & e\ \text{raises}\ v & \verb|e\raisesUp[\Sigma] v| \\
            e\raisesUp[\Sigma](c) v        & e\ \text{raises}\ v\ \text{with cost}\ c & \verb|e\raisesUp[\Sigma](c) v| \\\\
            %
            e\hasResult[\Sigma] v    & e\ \text{evaluates or raises}\ v & \verb|e\hasResult[\Sigma] v| \\
            e\hasResult[\Sigma](c) v    & e\ \text{evaluates or raises}\ v\ \text{with cost}\ c  & \verb|e\hasResult[\Sigma](c) v|
        \end{array}
    \end{displaymath}

    \caption{Judgments for Dynamics}
    \label{fig:dynamics}
\end{figure}

\section*{Dynamics}
\label{sec:dynamics}

The dynamics of a language determines how its constructs are to be executed, or evaluated.  It is given by a \emph{state transition system}, which consists of a collection of execution states, some of which are deemed initial, some of which are deemed final, and a relation between states that expresses ``one step'' of execution.  In principle a state can be anything at all, but in practice states are constructed from (closed, well-formed) pieces of syntax, perhaps involving symbols, and sometimes involving auxiliary data structures such as memories.  These considerations lead to the judgment forms given in Figure~\ref{fig:dynamics}.

There are, invariably, unlabelled forms of transition that may, or may not, be indexed by a signature defining the symbols that are available within the state.  In some cases labelled forms of transition are further indexed by an \emph{action} that specifies the effect of a step of execution.  Generally speaking, such transitions are not execution steps \textit{per se}, but are rather potential steps that may be taken in conjunction with another transition undertaking a ``complementary'' action.

Occasionally it is helpful to introduce evaluation relations for expressions and execution relations for commands that consolidate multiple individual steps of execution to a final state.  These are also specified in Figure~\ref{fig:dynamics}.  In most situations these relations are secondary to the transition systems that define a dynamics, but in some situations they may be defined directly, without or without reference to a transition system.


% \section*{Equality}
% \label{sec:equality}

% \subsection*{Formal Equality}
% \label{sec:equality-formal}

% \subsection*{Semantic Equality}
% \label{sec:equality-semantic}

\end{document}

